\question 

\begin{parts}

\part Let $X$ be the subspace of $\mathbb{R}^2$ consisting of the horizontal
segment $[0,1] \times \{0\}$ together with all the vertical segments $\{r\}
\times [0, 1-r]$ for $r$ a rational number in $[0, 1]$. Show that $X$
deformation retracts to any point in the segment $[0, 1] \times \{0\}$, but not 
to any other point.

\begin{solution}
%
The existence of a deformation retract from $X$ to any point $(r,0)$ is obvious:
Compose the map that shrinks the interval $[0,1] \times \{0\}$ to $(r,0)$ with
the map that shrinks the interval $\{r\} \times [0,1-r]$ to $(r,0)$.

Now, consider an arbitrary point $(r,s)$ with $s > 0$ and hence $r \in
\mathbb{Q}$. Suppose $X$ deformation retracts to $(r,s)$. Choose a neighborhood
$U$ of $(r,s)$ small enough to be disjoint from the ``base,'' i.e., containing
no points for the form $(\overline{r}, 0)$. Let $V \subseteq U$ as in Exercise
5, so $f:V \times I \rightarrow U$ is a homotopy from the inclusion map $V
\hookrightarrow U$ to the constant map sending all of $V$ to $(r,s)$.

We observe that if the inclusion $V \hookrightarrow U$ is nullhomotopic, then
$V$ must be path-connected, for $t \mapsto f_t(x)$ gives a path connecting $x$ 
to $x_0 = f_1(x)$ (where $x_0$ is independent of $x$). But obviously the $V$ 
we have just described is not path-connected.
    
\end{solution}

\part Let $Y$ be the subspace of $\mathbb{R}^2$ that is the union of an infinite 
number of copies of $X$ arranged as in the figure (in the book). Show that $Y$ 
is contractible but does not deformation retract onto any point.

% \begin{solution}
    
% \end{solution}

\part Let $Z$ be the zigzag subspace of $Y$ homeomorphic to $\mathbb{R}$
indicated by the heavier line. Show that there is a deformation retraction in
the weak sense of $Y$ onto $Z$, but no true deformation retraction.

% \begin{solution}
    
% \end{solution}

\end{parts}